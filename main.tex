\documentclass{beamer}
\usetheme{ucl}

\newcommand\hmmax{0}
\newcommand\bmmax{0}

\usepackage{graphicx}%
\usepackage{multirow}%
\usepackage{amsmath,amssymb,amsfonts}%
%\usepackage{amsthm}%
\usepackage{stmaryrd}
\usepackage{mathrsfs}%
\usepackage[title]{appendix}%
\usepackage{xcolor}%
\usepackage{textcomp}%
\usepackage{manyfoot}%
\usepackage{booktabs}%
\usepackage{algorithm}%
\usepackage{algorithmicx}%
\usepackage{algpseudocode}%
\usepackage{listings}%
\usepackage{orcidlink}
\usepackage{comment}
%%%%
\usepackage{mathtools}  % For cases
%\usepackage{thm-restate}
%\usepackage{mathrsfs}
%\usepackage[inline]{enumitem} %For enum*

\usepackage{mathbbol}
%\usepackage{tikz-cd} % For tikz art
%\usepackage{tikzit}
%\usepackage{ebproof}
\usepackage{bussproofs}
\usepackage{proof}


%%%%%%%%%%%%%%%%%%%%%%%%%%%%%%%%%%%%%%%%%%%%%%%%%%%%%%
\input{macro}


\setbeamersize{description width=2em}
%%% Remove nav symbols (and shift any logo down to corner)
\setbeamertemplate{navigation symbols}{\vspace{-2ex}}
\setbeamertemplate{footline}[author title date]
\setbeamercolor{banner}{bg=darkpurple}
\useinnertheme{blockborder}

\graphicspath{{./images/} }
\title[Translations between bases in B-eS]{Translations between bases in Base-extension Semantics}
%\author{\texorpdfstring{Yll Buzoku\newline\url{y.buzoku@ucl.ac.uk}}{https://www.homepages.ucl.ac.uk/~zcapybu/}}
\author{Yll Buzoku}
\institute[UCL]{%
  Department of Computer Science \\ %
  University College London
}
\date{December 11, 2024}

\AtBeginSection[]
{
  \begin{frame}
    \frametitle{Presentation root directory}
    \tableofcontents[currentsection]
  \end{frame}
}

\begin{document}
%%%%%%%%%%%%%%%%%%%%%%%%%%%%%%%%%%%%%%%%%%%%%%%%%%%%%%%%
%%%%%%%%%%%%%%%%%%%%%%%%%%%%%%%%%%%%%%%%%%%%%%%%%%%%%%%%
\begin{frame}
\titlepage
\end{frame}
%%%%%%%%%%%%%%%%%%%%%%%%%%%%%%%%%%%%%%%%%%%%%%%%%%%%%%%%
%%%%%%%%%%%%%%%%%%%%%%%%%%%%%%%%%%%%%%%%%%%%%%%%%%%%%%%%
\section*{Goals for this presentation}
\begin{frame}{Goals for this talk}
\begin{itemize}
\item Introduce the concept of bases, atomic rules and inference figures
\item Introduce two notions of atomic derivability
\item Show how one may relate these notions of atomic derivability
\end{itemize}
\end{frame}
%%%%%%%%%%%%%%%%%%%%%%%%%%%%%%%%%%%%%%%%%%%%%%%%%%%%%%%%
%%%%%%%%%%%%%%%%%%%%%%%%%%%%%%%%%%%%%%%%%%%%%%%%%%%%%%%%
%\begin{frame}{Presentation root directory}
%\tableofcontents
%\end{frame}
%%%%%%%%%%%%%%%%%%%%%%%%%%%%%%%%%%%%%%%%%%%%%%%%%%%%%%%%
%%%%%%%%%%%%%%%%%%%%%%%%%%%%%%%%%%%%%%%%%%%%%%%%%%%%%%%%
\section{Bases and atomic rules}
\begin{frame}{Notation}
	\begin{itemize}%[label={-}]
		\item $\At$ represents a fixed, countably infinite set of propositional atoms. 
		\item Lower case latin letters represent propositional atoms.
		\item Upper case latin letters represent finite multisets of propositional atoms. 
		%\item Atomic multiset is taken to mean multiset of propositional atoms.
		\item The sum of two multisets $\at{P}$ and $\at{Q}$ is denoted $\at{P}\msetsum\at{Q}$. 
	\end{itemize}
\end{frame}
%%%%%%%%%%%%%%%%%%%%%%%%%%%%%%%%%%%%%%%%%%%%%%%%%%%%%%%%
\begin{frame}{Atomic rules}
Atomic rules take the form: \newline 
\begin{center}
	$(P_1\Rightarrow p_1),\dots, (P_n\Rightarrow p_n)\Rightarrow p$	
\end{center}
\end{frame}
%%%%%%%%%%%%%%%%%%%%%%%%%%%%%%%%%%%%%%%%%%%%%%%%%%%%%%%%
\begin{frame}{Atomic rules}
Pictorially, we can represent this as: \newline 
\begin{center}
	\begin{prooftree}
		\AxiomC{$[P_1]$}
		\noLine
		\UnaryInfC{$p_1$}
		\AxiomC{$\dots$}
		\AxiomC{$[P_n]$}
		\noLine
		\UnaryInfC{$p_n$}
		\TrinaryInfC{$p$}
	\end{prooftree}
	\pause
	\vspace{1.8cm}
	\emph{Any idea what these figures are supposed to look like?}\newline
	\pause
	\textbf{Natural deduction!}
\end{center}
\end{frame}
%%%%%%%%%%%%%%%%%%%%%%%%%%%%%%%%%%%%%%%%%%%%%%%%%%%%%%%%
\begin{frame}{Contextual atomic rules}
	Contextual atomic rules are rules which may have contextual brackets distributed across them.
\end{frame}
%%%%%%%%%%%%%%%%%%%%%%%%%%%%%%%%%%%%%%%%%%%%%%%%%%%%%%%%
\begin{frame}{Contextual atomic rules}
Pictorially, we can represent such rules as: \newline 

\begin{prooftree}
	\AxiomC{$\left\{\raisebox{-0.5em}{$\deduce{p_{1_1} \quad \ldots \quad p_{1_{l_1}}}{[P_{1_1}] \hspace{2.8em} [P_{1_{l_n}}]}$}\right\}$}
	\AxiomC{$\ldots$}
	\AxiomC{$\left\{\raisebox{-0.5em}{$\deduce{p_{n_1} \quad \ldots \quad p_{n_{l_n}}}{[P_{n_1}] \hspace{2.8em} [P_{n_{l_n}}]}$}\right\}$}
	\TrinaryInfC{$p$}
\end{prooftree}
\pause
or linearly, like this:\newline
\begin{center}
	$\openaddrule (P_{1_i}\Rightarrow p_{1_i})\closeaddrule^{l_1}_{i=1}, \dots, \openaddrule (P_{n_i}\Rightarrow p_{n_i})\closeaddrule^{l_n}_{i=1}\Rightarrow p$
\end{center}
\end{frame}
%%%%%%%%%%%%%%%%%%%%%%%%%%%%%%%%%%%%%%%%%%%%%%%%%%%%%%%%
\begin{frame}{Examples of atomic rules}
	\begin{itemize}
		\item $\openaddrule \Rightarrow a\closeaddrule\Rightarrow c$
		\item $\openaddrule \Rightarrow a\closeaddrule, \openaddrule (c\msetsum d \Rightarrow e), (f\Rightarrow g)\closeaddrule\Rightarrow q$ 
		\item $\openaddrule a\Rightarrow b\closeaddrule, \openaddrule\Rightarrow e \closeaddrule, (f\Rightarrow g)\Rightarrow q$
	\end{itemize}
	\vspace{0.5cm}
	\begin{itemize}
		\item $(\Rightarrow a) \Rightarrow c$
		\item $(\Rightarrow b) \Rightarrow c$
		\item $(\Rightarrow c), (a \Rightarrow p), (b\Rightarrow p)\Rightarrow p$
	\end{itemize}
\end{frame}
%%%%%%%%%%%%%%%%%%%%%%%%%%%%%%%%%%%%%%%%%%%%%%%%%%%%%%%%
\begin{frame}{Bases}
	\begin{definition}[Base]
		A base is a set of atomic rules.\newline
		\pause
		A base is said to be contextual if it contains contextual rules.\\
		Else it is said to be context-free.
	\end{definition}
	\pause
	\begin{itemize}
		\item $\{((\Rightarrow a) \Rightarrow c), ((\Rightarrow b) \Rightarrow c), ((\Rightarrow c), (a \Rightarrow p), (b\Rightarrow p)\Rightarrow p)\}$
		%
		\item $\{(\openaddrule \Rightarrow a\closeaddrule\Rightarrow c), (\openaddrule a\Rightarrow c\closeaddrule\Rightarrow b),(\openaddrule\Rightarrow a\closeaddrule,\openaddrule\Rightarrow a\closeaddrule\Rightarrow a)\}$
		%
		\item $\{(\openaddrule \Rightarrow a\closeaddrule\Rightarrow c), (\openaddrule a\Rightarrow c\closeaddrule\Rightarrow b),(\openaddrule\Rightarrow d\closeaddrule,\openaddrule\Rightarrow d\closeaddrule\Rightarrow a),(\openaddrule\Rightarrow a\closeaddrule\Rightarrow d)\}$
	\end{itemize}
\end{frame}
%%%%%%%%%%%%%%%%%%%%%%%%%%%%%%%%%%%%%%%%%%%%%%%%%%%%%%%%
%%%%%%%%%%%%%%%%%%%%%%%%%%%%%%%%%%%%%%%%%%%%%%%%%%%%%%%%
\section{Atomic derivability}
\begin{frame}
\begin{definition}[Context-free atomic derivability $\deriveBaseIPL{\baseB}$]
The relation of derivability in a context-free base $\baseB$, is defined as so:
\begin{itemize}
    \item[Ref] $S\deriveBaseIPL{\baseB}p$ if $p\in S$.
    \item[App] For $((P_1\Rightarrow p_1), \dots, (P_n\Rightarrow p_n)\Rightarrow q) \in \baseB$ and $S\msetsum P_i\deriveBaseIPL{\baseB}p_i$ for each $i\in \{1,\dots,n\}$ then $S\deriveBaseIPL{\baseB}q$.
\end{itemize}
\end{definition}
\pause
\begin{definition}[Contextual atomic derivability $\deriveBaseM{\baseB}$]
The relation of derivability in a contextual base $\baseB$, is defined as so:
\begin{itemize}
    \item[Ref] $p\deriveBaseM{\baseB}p$.
    \item[App] For $(\openaddrule(P_{1_i}\Rightarrow p_{1_{i}})\Closeaddrule{i=1}{l_1},\dots,\openaddrule (P_{n_i}\Rightarrow p_{n_{i}})\Closeaddrule{i=1}{l_n}\Rightarrow q) \in \baseB$ and $C_i\msetsum P_{i_j}\deriveBaseM{\baseB}p_{i_j}$ for each $i\in \{1,\dots,n\}$ and $j \in \{1,\dots,l_i\}$ then $C_1\msetsum\dots\msetsum C_n\deriveBaseM{\baseB}q$.
\end{itemize}
\end{definition}

\end{frame}
%%%%%%%%%%%%%%%%%%%%%%%%%%%%%%%%%%%%%%%%%%%%%%%%%%%%%%%%
\begin{frame}{Example derivations}
	\begin{example}
		Let $\baseB=\{(\Rightarrow a), ((\Rightarrow a),(\Rightarrow b) \Rightarrow c)\}$
		\begin{prooftree}
			\AxiomC{}
			\RightLabel{Ref}
			\UnaryInfC{$b\deriveBaseIPL{\baseB}b$}
			\AxiomC{}
			\RightLabel{$\Rightarrow a$}
			\UnaryInfC{$\deriveBaseIPL{\baseB}a$}
			\RightLabel{$(\Rightarrow a),(\Rightarrow b)\Rightarrow c$}
			\BinaryInfC{$b\deriveBaseIPL{\baseB}c$}
		\end{prooftree}
	\end{example}
\end{frame}
%%%%%%%%%%%%%%%%%%%%%%%%%%%%%%%%%%%%%%%%%%%%%%%%%%%%%%%%
\begin{frame}{Example derivations}
	\begin{example}
		Let $\baseB=\{(\Rightarrow a),(\openaddrule \Rightarrow a\closeaddrule,\openaddrule \Rightarrow b\closeaddrule\Rightarrow c)\}$
		\begin{prooftree}
			\AxiomC{}
			\RightLabel{Ref}
			\UnaryInfC{$b\deriveBaseM{\baseB}b$}
			\AxiomC{}
			\RightLabel{$\Rightarrow a$}
			\UnaryInfC{$\deriveBaseM{\baseB}a$}
			\RightLabel{$\openaddrule \Rightarrow a \closeaddrule,\openaddrule \Rightarrow b\closeaddrule\Rightarrow c$}
			\BinaryInfC{$b\deriveBaseM{\baseB}c$}
		\end{prooftree}
	\end{example}
\end{frame}
%%%%%%%%%%%%%%%%%%%%%%%%%%%%%%%%%%%%%%%%%%%%%%%%%%%%%%%%
\begin{frame}{Example derivations}
	\begin{example}
		By the definition of $\deriveBaseM{\baseB}$, deriving $a$ from $a\msetsum a$ in the empty base is not possible, i.e. $a\msetsum a\deriveBaseM{\emptybase}a$ is not possible. But $a\msetsum a\deriveBaseIPL{\emptybase}a$ is possible.
	\end{example}
	\pause
	\begin{example}
		Let $\baseB=\{(\openaddrule \Rightarrow a\closeaddrule\Rightarrow c), (\openaddrule a\Rightarrow c\closeaddrule\Rightarrow b)\}$
		\begin{prooftree}
			\AxiomC{$\times$}
			\UnaryInfC{$a\msetsum a\deriveBaseM{\baseB}a$}
			\RightLabel{$\openaddrule \Rightarrow a\closeaddrule\Rightarrow c$}
			\UnaryInfC{$a\msetsum a\deriveBaseM{\baseB}c$}
			\RightLabel{$\openaddrule a\Rightarrow c\closeaddrule\Rightarrow b$}
			\UnaryInfC{$a\deriveBaseM{\baseB}b$}
		\end{prooftree}
		We see that in this base, $a$ is not derivable from $a\msetsum a$.
	\end{example}
\end{frame}
%%%%%%%%%%%%%%%%%%%%%%%%%%%%%%%%%%%%%%%%%%%%%%%%%%%%%%%%
\begin{frame}{Example derivations}
	\begin{example}[A possible fix]
		Let $\illipl{\baseB}=\{((\Rightarrow a)\Rightarrow c), ((a\Rightarrow c)\Rightarrow b)\}$
		\begin{prooftree}
			\AxiomC{}
			\RightLabel{Ref}
			\UnaryInfC{$a\msetsum a\deriveBaseIPL{\illipl{\baseB}}a$}
			\RightLabel{$(\Rightarrow a)\Rightarrow c$}
			\UnaryInfC{$a\msetsum a\deriveBaseIPL{\illipl{\baseB}}c$}
			\RightLabel{$(a\Rightarrow c)\Rightarrow b$}
			\UnaryInfC{$a\deriveBaseIPL{\illipl{\baseB}}b$}
		\end{prooftree}
	\end{example}
\end{frame}
%%%%%%%%%%%%%%%%%%%%%%%%%%%%%%%%%%%%%%%%%%%%%%%%%%%%%%%%
\begin{frame}{Example derivations}
	\begin{example}[Another possible fix]
		Let $\baseC=\{(\openaddrule \Rightarrow a\closeaddrule\Rightarrow c), (\openaddrule a\Rightarrow c\closeaddrule\Rightarrow b),(\openaddrule\Rightarrow a\closeaddrule,\openaddrule\Rightarrow a\closeaddrule\Rightarrow a)\}$
		\begin{prooftree}
			\AxiomC{}
			\RightLabel{Ref}
			\UnaryInfC{$a\deriveBaseM{\baseC}a$}			
			\AxiomC{}
			\RightLabel{Ref}
			\UnaryInfC{$a\deriveBaseM{\baseC}a$}
			\RightLabel{$\openaddrule\Rightarrow a\closeaddrule,\openaddrule\Rightarrow a\closeaddrule\Rightarrow a$}
			\BinaryInfC{$a\msetsum a\deriveBaseM{\baseC}a$}
			\RightLabel{$\openaddrule \Rightarrow a\closeaddrule\Rightarrow c$}
			\UnaryInfC{$a\msetsum a\deriveBaseM{\baseC}c$}
			\RightLabel{$\openaddrule a\Rightarrow c\closeaddrule\Rightarrow b$}
			\UnaryInfC{$a\deriveBaseM{\baseC}b$}
		\end{prooftree}
		Note that $\baseC\supset\baseB$\hspace{5pt}!!
	\end{example}
\end{frame}
%%%%%%%%%%%%%%%%%%%%%%%%%%%%%%%%%%%%%%%%%%%%%%%%%%%%%%%%
\begin{frame}{Example derivations}
	\begin{example}[A very interesting derivation]
		Let $\baseB$ be the following set of rules: \\$\{
		(\openaddrule\Rightarrow b\closeaddrule\Rightarrow c),
		(\openaddrule\Rightarrow a\closeaddrule,\openaddrule b\Rightarrow c\closeaddrule\Rightarrow d),
		(\openaddrule\Rightarrow d, \Rightarrow a \closeaddrule\Rightarrow e),
		(\openaddrule a\Rightarrow e\closeaddrule\Rightarrow f)
		\}$
		\begin{prooftree}
			\AxiomC{}
			\RightLabel{Ref}
			\UnaryInfC{$a\deriveBaseM{\baseB}a$}			
			\AxiomC{}
			\RightLabel{Ref}
			\UnaryInfC{$b\deriveBaseM{\baseB}b$}
			\RightLabel{$\openaddrule\Rightarrow b\closeaddrule\Rightarrow c$}
			\UnaryInfC{$b\deriveBaseM{\baseB}c$}
			\RightLabel{$\openaddrule\Rightarrow a\closeaddrule,\openaddrule b\Rightarrow c\closeaddrule\Rightarrow d$}
			\BinaryInfC{$a\deriveBaseM{\baseB}d$}
			\AxiomC{}
			\RightLabel{Ref}
			\UnaryInfC{$a\deriveBaseM{\baseB}a$}
			\RightLabel{$\openaddrule\Rightarrow d, \Rightarrow a \closeaddrule\Rightarrow e$}
			\BinaryInfC{$a\deriveBaseM{\baseB}e$}
			\RightLabel{$\openaddrule a\Rightarrow e\closeaddrule\Rightarrow f$}
			\UnaryInfC{$\deriveBaseM{\baseB}f$}
		\end{prooftree}
	\end{example}
\end{frame}
%%%%%%%%%%%%%%%%%%%%%%%%%%%%%%%%%%%%%%%%%%%%%%%%%%%%%%%%
%\begin{frame}{Example derivations}
%	\begin{example}[A live complicated derivation!]
%		Consider a base $\baseB$ with only the following rules:
%		\begin{itemize}
%			\item $\openaddrule\Rightarrow c\closeaddrule,\openaddrule\Rightarrow y\closeaddrule\Rightarrow f$
%			\item $\Rightarrow z$
%			\item $\openaddrule \Rightarrow a\closeaddrule,\openaddrule \Rightarrow b\closeaddrule,\openaddrule\Rightarrow z\closeaddrule\Rightarrow h$
%			\item $\openaddrule\Rightarrow h\closeaddrule\Rightarrow g$			\item $\openaddrule\Rightarrow x\closeaddrule,\openaddrule\Rightarrow g\closeaddrule\Rightarrow e$
%			\item $\openaddrule x\Rightarrow e\closeaddrule,\openaddrule y\Rightarrow f\closeaddrule\Rightarrow d$
%		\end{itemize}
%		is there a derivation of $d$ from the multiset $a\msetsum b\msetsum c$ in $\baseB$?
%		\begin{prooftree}
%			\AxiomC{}
%			\RightLabel{Ref}
%			\UnaryInfC{$f\deriveBaseM{\baseB}f$}
%			\AxiomC{}
%			\RightLabel{Ref}
%			\UnaryInfC{$c\deriveBaseM{\baseB}c$}
%			\RightLabel{$\openaddrule\Rightarrow c\closeaddrule,\openaddrule\Rightarrow y\closeaddrule\Rightarrow f$}
%			\BinaryInfC{$c,y\deriveBaseM{\baseB}f$}
%			\AxiomC{}
%			\RightLabel{$\Rightarrow z$}
%			\UnaryInfC{$\deriveBaseM{\baseB}z$}
%			\AxiomC{}
%			\RightLabel{Inf}
%			\UnaryInfC{$b\deriveBaseM{\baseB}b$}
%			\AxiomC{}
%			\RightLabel{Inf}
%			\UnaryInfC{$a\deriveBaseM{\baseB}a$}
%			\RightLabel{$\openaddrule \Rightarrow a\closeaddrule,\openaddrule \Rightarrow b\closeaddrule,\openaddrule\Rightarrow z\closeaddrule\Rightarrow h$}
%			\TrinaryInfC{$a,b,\deriveBaseM{\baseB}h$}
%			\RightLabel{$\openaddrule\Rightarrow h\closeaddrule\Rightarrow g$}
%			\UnaryInfC{$a,b\deriveBaseM{\baseB}g$}
%			\AxiomC{}
%			\RightLabel{Ref}
%			\UnaryInfC{$x\deriveBaseM{\baseB}x$}
%			\RightLabel{$\openaddrule\Rightarrow x\closeaddrule,\openaddrule\Rightarrow g\closeaddrule\Rightarrow e$}
%			\BinaryInfC{$a,b,x\deriveBaseM{\baseB}e$}
%			\RightLabel{$\openaddrule x\Rightarrow e\closeaddrule,\openaddrule y\Rightarrow f\closeaddrule\Rightarrow d$}
%			\BinaryInfC{$a,b,c\deriveBaseM{\baseB}b$}
%		\end{prooftree}
%	\end{example}
%	\pause 
%	\emph{Disclaimer: The fact I couldn't fit this derivation on the slide has nothing to do with why we are doing this derivation live :-)}
%\end{frame}
%%%%%%%%%%%%%%%%%%%%%%%%%%%%%%%%%%%%%%%%%%%%%%%%%%%%%%%%
%%%%%%%%%%%%%%%%%%%%%%%%%%%%%%%%%%%%%%%%%%%%%%%%%%%%%%%%
\section{Comparing relations}
\begin{frame}{Comparing relations}
\begin{definition}[Structural rules]
	We define two rules:
	\begin{itemize}
		\item $\text{Wk}^p_q = (\openaddrule \Rightarrow p\closeaddrule, \openaddrule \Rightarrow q \closeaddrule\Rightarrow q)$
		%
		\item $\text{Ctn}^p_q = (\openaddrule \Rightarrow p \closeaddrule, \openaddrule p\msetsum p\Rightarrow q\closeaddrule\Rightarrow q)$
	\end{itemize}
\end{definition}
\end{frame}
%%%%%%%%%%%%%%%%%%%%%%%%%%%%%%%%%%%%%%%%%%%%%%%%%%%%%%%%
\begin{frame}{Mappings between bases}
	\pause
	\begin{definition}
		Let $\baseB$ be a context-free base. We define structural contextualisation of that base $\iplill{\baseB}$ as follows: 
		\begin{minipage}{\textwidth}\scriptsize
		\begin{align*}
		\iplill{\baseB} &= \,\{ \openaddrule P_1\Rightarrow p_1\closeaddrule, \dots, \openaddrule P_n\Rightarrow p_n\closeaddrule\Rightarrow q \, | \, ((P_1\Rightarrow p_1), \dots, (P_n\Rightarrow p_n)\Rightarrow q) \in\baseB \}\\ 
		& \cup \{\text{Wk}^p_q, \text{Ctn}^p_q \,|\, \forall p, q \in \At\}
		\end{align*}
	\end{minipage}
	\end{definition}
	\pause
	We define the decontextualisation of a contextual base as:
	%\begin{minipage}{\textwidth}\scriptsize
	\begin{align*}
	\illipl{\baseB} = \{& (P_{1_1}\Rightarrow p_{1_{1}}),\dots,(P_{n_{l_n}}\Rightarrow p_{n_{l_n}})\Rightarrow q \\
	&|\, (\openaddrule P_{1_i}\Rightarrow p_{1_{i}}\Closeaddrule{i=1}{l_1},\dots,\openaddrule P_{n_i}\Rightarrow p_{n_{i}} \Closeaddrule{i=1}{l_n}\Rightarrow q) \in \baseB\}
	\end{align*}
	%\end{minipage}
\end{frame}
%%%%%%%%%%%%%%%%%%%%%%%%%%%%%%%%%%%%%%%%%%%%%%%%%%%%%%%%
\begin{frame}{Properties of these mappings}
	Let $\baseB$ be a context-free base. Then the following hold:
	\begin{itemize}
		\item $\illipl{\iplill{\baseB}}\baseGeq \baseB$.
		\pause
		\item $\iplill{\illipl{\iplill{\baseB}}}= \iplill{\baseB}$.
		\pause 
		\item For all $\baseC\baseGeq\baseB$ we have that $\iplill{\baseC}\baseGeq\iplill{\baseB}$.
		\pause
		\item For all $\baseC\baseGeq\iplill{\baseB}$ there exists an extension $\baseX\baseGeq\baseB$ such that $\iplill{\baseX}=\baseC$. 
	\end{itemize}

\end{frame}
%%%%%%%%%%%%%%%%%%%%%%%%%%%%%%%%%%%%%%%%%%%%%%%%%%%%%%%%
\begin{frame}{Structural admissibility in a structurally contextualised base}
	\begin{lemma}
		Suppose $L\deriveBaseM{\iplill{\baseB}}p$ holds. Then $R\msetsum L\deriveBaseM{\iplill{\baseB}}p$ also holds for any atomic multiset $R$.
	\end{lemma}
\end{frame}
%%%%%%%%%%%%%%%%%%%%%%%%%%%%%%%%%%%%%%%%%%%%%%%%%%%%%%%%
\begin{frame}
	\begin{proof}
		Let $R = \{r_1,\dots , r_n\}$ for some $n$. Then we have that we can effectively weaken $S$ away as follows:

		\begin{minipage}{0.5\textwidth}\scriptsize
		\begin{prooftree}
		\AxiomC{}
		\RightLabel{Ref}
		\UnaryInfC{$r_1\deriveBaseM{\iplill{\baseB}}r_1$}
		\AxiomC{}
		\RightLabel{Ref}
		\UnaryInfC{$r_{n-1}\deriveBaseM{\iplill{\baseB}}r_{n-1}$}
		\AxiomC{}
		\RightLabel{Ref}
		\UnaryInfC{$r_n\deriveBaseM{\iplill{\baseB}}r_n$}
		\AxiomC{$L\deriveBaseM{\iplill{\baseB}}p$}
		\RightLabel{$\text{Wk}^{r_n}_p$}
		\BinaryInfC{$r_n\msetsum L \deriveBaseM{\iplill{\baseB}}p$}
		\RightLabel{$\text{Wk}^{r_{n-1}}_p$}
		\BinaryInfC{\vdots}
		\noLine
		\UnaryInfC{$r_2\msetsum\dots\msetsum r_n\msetsum L \deriveBaseM{\iplill{\baseB}}p$}
		\RightLabel{$\text{Wk}^{r_1}_p$}
		\BinaryInfC{$R\msetsum L \deriveBaseM{\iplill{\baseB}}p$}
		\end{prooftree}
		\end{minipage}
	\end{proof}
\end{frame}
%%%%%%%%%%%%%%%%%%%%%%%%%%%%%%%%%%%%%%%%%%%%%%%%%%%%%%%%
\begin{frame}{Structural admissibility in a structurally contextualised base}
	\begin{lemma}
		Suppose $R\msetsum R \deriveBaseM{\iplill{\baseB}}p$ holds. Then $R\deriveBaseM{\iplill{\baseB}}p$ also holds.
	\end{lemma}
	\pause
	\begin{corollary}
		For arbitrary $m\geq 1$, if $R^m\deriveBaseM{\iplill{\baseB}}p$ then $R\deriveBaseM{\iplill{\baseB}}p$.
	\end{corollary}
\end{frame}
%%%%%%%%%%%%%%%%%%%%%%%%%%%%%%%%%%%%%%%%%%%%%%%%%%%%%%%%
\begin{frame}
	\begin{proof}
		Let $R = \{r_1, \dots , r_n\}$ for some $n$. Then we have that we can effectively contract on $R$ as follows:

		\begin{minipage}{0.5\textwidth}\scriptsize
			\begin{prooftree}
				\AxiomC{}
				\RightLabel{Ref}
				\UnaryInfC{$r_1\deriveBaseM{\iplill{\baseB}}r_1$}
				\AxiomC{}
				\RightLabel{Ref}
				\UnaryInfC{$r_{n-1}\deriveBaseM{\iplill{\baseB}}r_{n-1}$}
				\AxiomC{}
				\RightLabel{Ref}
				\UnaryInfC{$r_n\deriveBaseM{\iplill{\baseB}}r_n$}
				\AxiomC{$R\msetsum R\deriveBaseM{\iplill{\baseB}}p$}
				\RightLabel{$\text{Ctn}^{r_n}_p$}
				\BinaryInfC{$r_1\msetsum\dots\msetsum r_{n-1}\msetsum R \deriveBaseM{\iplill{\baseB}}p$}
				\RightLabel{$\text{Ctn}^{r_{n-1}}_p$}
				\BinaryInfC{\vdots}
				\noLine
				\UnaryInfC{$r_1\msetsum R \deriveBaseM{\iplill{\baseB}}p$}
				\RightLabel{$\text{Ctn}^{r_1}_p$}
				\BinaryInfC{$R\deriveBaseM{\iplill{\baseB}}p$}
			\end{prooftree}
		\end{minipage}
	\end{proof}
\end{frame}
%%%%%%%%%%%%%%%%%%%%%%%%%%%%%%%%%%%%%%%%%%%%%%%%%%%%%%%%
\begin{frame}{Key results under this base translation}
	\begin{itemize}
		\item If $R\deriveBaseIPL{\baseB}p$ then $R\deriveBaseM{\iplill{\baseB}}p$.
		\vspace{0.3cm}
		\pause
		\item $R\deriveBaseM{\iplill{\baseB}}p$ iff for all bases $\baseX \baseGeq \iplill{\baseB}$ where for each $r \in R$ we have $\deriveBaseM{\baseX}r$ then it follows that $\deriveBaseM{\baseX}p$.
		\vspace{0.3cm}
		\pause
		\item $R\deriveBaseM{\iplill{\baseB}}p$ iff $\bang R \suppM{\iplill{\baseB}}{\emptymultiset}p$
		\vspace{0.3cm}
		%\pause
		%\item $\suppM{\iplill{\baseB}}{R}p$ iff $\bang R\suppM{\iplill{\baseB}}{\emptymultiset}p$
	\end{itemize}
\end{frame}
%%%%%%%%%%%%%%%%%%%%%%%%%%%%%%%%%%%%%%%%%%%%%%%%%%%%%%%%

\begin{frame}{Thank you!}
	\begin{center}
		\textbf{Thank you for listening! \\ 
		Comments? Observations? Please ask and/or feel free to \\
		email me at y.buzoku@ucl.ac.uk}
	\end{center}
%\begin{figure}
%	\begin{center}
%	  \includegraphics[width=\textwidth]{dosthanks2.png}
%	  \caption{Thank you from DOS!\,:D}
%	\end{center}
%  \end{figure}
\end{frame}
%%%%%%%%%%%%%%%%%%%%%%%%%%%%%%%%%%%%%%%%%%%%%%%%%%%%%%%%
\begin{frame}[allowframebreaks]
	\frametitle{References}
	\nocite{*}
	\bibliographystyle{amsalpha}
	\bibliography{./refs/refs.bib}
\end{frame}
%%%%%%%%%%%%%%%%%%%%%%%%%%%%%%%%%%%%%%%%%%%%%%%%%%%%%%%%
\end{document}
